\section{Introduction to Dice}

\noindent
Dice are the heart of uncertainty.  
\textit{Veni Alea} uses the standard polyhedral set:
\texttt{d4, d6, d8, d10, d12, d20}, along with several extended forms for advanced rolls:
\texttt{d15, d30, d50, d100}.  

These dice measure effort, risk, and consequence — the rhythm of fortune in play.

\subsection*{The Dice Pool}

Each character begins with a \textbf{dice pool} — a limited reserve of dice representing their capacity for action, reaction, and recovery.  
Your pool size and the colour of dice available depend on your \textbf{class} and \textbf{level}.  
As you advance, your pool expands, granting access to stronger or more specialised dice.

\begin{itemize}
  \item \textbf{In combat:} Dice are spent for attacks, defences, or special abilities.  
  Once used, they are temporarily exhausted until recovered.
  \item \textbf{Between combats:} Dice recover at a fixed \textbf{recovery rate} — usually several per short rest, or the full pool after a long rest.
  \item \textbf{Outside combat:} Dice may be \textbf{sacrificed} to perform heroic actions, ensure narrative success, or upgrade later rolls.  
  Such sacrifices recover only after a full day.
\end{itemize}

\subsection*{Modifiers and Upgrading}

Every die roll can be modified by:
\begin{itemize}
  \item \textbf{Skill or Attribute bonuses} — reflecting a character’s innate or trained ability.
  \item \textbf{Circumstance} — environmental advantage or disadvantage.
  \item \textbf{Opposition} — the quality or strength of the resisting force.
\end{itemize}

When these factors combine in a character’s favour, the die may be \textbf{upgraded} to a larger die type.  
This is called the \textbf{Chain of Dice}.

\subsection*{The Chain of Dice}

\noindent
\texttt{d2 → d4 → d6 → d8 → d10 → d12 → d15 → d20 → d30 → d50 → d100}

\noindent
Each step up represents a sharper edge of potential — more power, but greater variance.  
When instructed to “upgrade” a die, move one step along this chain.  
When “downgrading,” step back.  
If you exceed either end of the chain, you instead roll twice and take the higher (for upgrades) or lower (for downgrades).

\subsection*{Emulating Nonstandard Dice}

Most standard sets do not include \texttt{d15}, \texttt{d30}, or \texttt{d50}.  
They can be simulated using existing dice:

\begin{itemize}
  \item \textbf{d15} — roll a \texttt{d20} and reroll any result above 15.  
  \item \textbf{d30} — roll a \texttt{d20} and a \texttt{d10}.  
  Treat the \texttt{d10} as adding +10 if the \texttt{d20} shows 11–20.  
  (So 1–10 = 1–10, 11–20 = 11–20, 21–30 = 21–30.)  
  Alternatively, roll a \texttt{d6} × 5 for a coarser approximation.
  \item \textbf{d50} — roll a \texttt{d100} and divide by two, rounding up.  
  (1–2 → 1, 3–4 → 2, etc.)
\end{itemize}

The aim is not numerical purity but narrative consistency — a sense that stronger actions roll bigger dice, and that every escalation carries risk.
