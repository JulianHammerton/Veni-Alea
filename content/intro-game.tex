\section{Introduction to the Game}

\noindent
\textbf{Veni Alea} is a game of \textit{fate, chance, and consequence}.  
As a role-playing framework, the intent is for the players themselves to embody characters in a shared story.  The game master (GM) sets the scene, describes the world, and adjudicates outcomes, while players take on the roles of heroes navigating challenges, conflicts, and adventures.

The dice — the \textit{alea} — are the language of uncertainty.  They can be upgraded or sacrificed to reflect effort, risk, and fortune’s favour.  They can also be downgraded if circumstances turn against the heroes.  Every roll is a moment of tension: will the hero’s daring pay off, or will fate intervene?

The rhythms of each session might be:
\begin{itemize}
  \item \textbf{Role-play:} Characters speak, act, and decide; personality guides the hand that rolls.  
  \item \textbf{Resolution:} Impasses or conflicts arise, requiring dice rolls to determine outcomes.
  \item \textbf{Combat:} Players can be emperilled by their decisions and be forced to fight.
  \item \textbf{Narration:} The Game Master describes the results, weaving success and failure into the unfolding tale.
\end{itemize}

\medskip
\noindent
\textit{Veni Alea} provides a framework not to restrict imagination, but to \textbf{amplify} it.  
