\section{Introduction to Role-Playing}

Role-playing in \textit{Veni Alea} is conversation shaped by rules and tempered by uncertainty.  
The Game Master (GM) maintains the living world, while the players animate it through intent, reaction, and improvisation.  
Yet unlike most games of this kind, narrative control does not rest solely with the GM.

\subsection*{Distributed Scene Leadership}

Each new encounter or narrative turn begins when the GM delegates control of a \textbf{scene} to one of the players.  
That player becomes the \textbf{Scene Leader}, responsible for setting tone, pacing, and participation within the frame.  
The Scene Leader invites or “subs in” other players to take on characters—some recurring, some newly created, and some ephemeral to the moment.  
Through these shifting roles, the story becomes communal and polyphonic rather than hierarchical.

The GM retains the authority to:
\begin{itemize}
  \item resolve minor disputes or mechanical triggers (such as injuries, timing, or resource loss);
  \item introduce new non-player characters or information through cameo appearances;
  \item correct contradictions or rein in divergence that would break narrative cohesion.
\end{itemize}

Much of the GM’s influence is exercised indirectly—through hidden knowledge, unseen motives, or subtle steering that keeps the world internally consistent.  
Players, meanwhile, experience a balance of agency and surprise: they drive the story forward, but the ground beneath their feet is always shifting.

\subsection*{Scene Resolution and the Trinity of Dice}

When the outcome of a moment is uncertain, players collectively identify three defining \textbf{factors} that will determine how the scene might unravel.  
Each factor represents a different aspect of probability or consequence, forming a triad known as the \textbf{Trinity of Dice}.

\begin{enumerate}
  \item \textbf{Intent or Action} — What is being attempted? (e.g. bribing a guard, breaking a lock, persuading a rival)
  \item \textbf{Character or Opposition} — How resistant or predisposed is the other side?
  \item \textbf{Consequence or Repercussion} — What risks or ripple effects might follow?
\end{enumerate}

Each factor is represented by a single \textbf{d10}.  
Before rolling, the players and GM assign modifiers that reflect the scene’s context—skill, circumstance, and plausibility.  
Once agreed, the GM may subtly adjust these modifiers to maintain fairness or reflect hidden information.  
The dice are then rolled simultaneously.  
The highest adjusted die defines the outcome and its tone:

\begin{itemize}
  \item If \textbf{Intent} wins — the player’s action succeeds as envisioned.
  \item If \textbf{Opposition} wins — resistance prevails; the attempt fails or is complicated.
  \item If \textbf{Consequence} wins — the action occurs, but at a cost or with unintended fallout.
\end{itemize}

\subsection*{Example}

A player proposes to bribe a guard to gain access to a restricted vault.  
Three factors are declared:

\begin{itemize}
  \item \textbf{Intent:} The bribe itself — offering 20 gold for silence.  
  \item \textbf{Opposition:} The guard’s integrity — dishonest, drunk.  
  \item \textbf{Consequence:} The risk of discovery — punishment for dereliction of duty.
\end{itemize}

Modifiers are set:
\begin{itemize}
  \item Intent: +2 (plausible and well-funded)  
  \item Opposition: -2 (players targeted a corrupt guard)  
  \item Consequence: +4 (the vault is heavily guarded)
\end{itemize}

The three dice are rolled.  
If the guard’s die (Opposition) rolls highest, he refuses the bribe; perhaps even raises the alarm.  
If the player’s Intent die triumphs, the bribe is accepted and the scene moves forward.  
If Consequence wins, the guard takes the bribe—but later, another officer notices the infraction, and fallout ensues.

Through this system, conflict becomes a dialogue of probabilities, each scene a negotiation between will, risk, and fate.
